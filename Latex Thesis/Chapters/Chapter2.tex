\documentclass[../Thesis.tex]{subfiles}
\graphicspath{{\subfix{../figures/}}}
\begin{document}
\chapter{Ideer til hvad der skal laves}

Overall model for throughput of system. I.e. model the system as e.g. a system of queues and how much is produced at each step and this propagate. The important aspect is breakdown (extra processing time) and possibility of having to trowing out some production along the way, either due to error or some other (unforeseen) causes.

Need to investigate different ways of modelling this (starting with a simple system with no queuing, i.e. a single batch; this is what is done above). Discuss the pros and cons and how much information they preserve (aggregation models etc. may need to model so'me part of the system by throwing away)


\end{document}
