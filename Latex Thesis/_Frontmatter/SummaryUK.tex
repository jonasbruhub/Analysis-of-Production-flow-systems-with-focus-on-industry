\chapter{Summary (English)}

The goal of the thesis is to ...






\cite{An_effective_approach_for_causal_variables_analysis_in_diesel_engine_production_by_using_mutual_information_and_network_deconvolution}
The effective control of the power consistency, which is one of the most important quality indicators of diesel engine, plays a decisive role for improving the competitiveness of the products. The widely used sensors and other data acquisition equipment make the “data-driven quality control” become possible. However, how to determine the highly related parameters with the engine power from massive captured manufacturing data and effectively discriminated the direct and indirect dependencies between these variables are still challenging. This paper proposed a feature selection algorithm named NMI-ND which uses network deconvolution (ND) to infer causal correlations among various diesel engine manufacturing parameters from the observed correlations based on normalized mutual information (NMI). The proposed algorithm is thoroughly evaluated through the experimental study by comparing it with other representative feature selection algorithms. The comparison demonstrates that NMI-ND performs better in both effectiveness and efficiency.



\cite{Network-deconvolution-as-a-general-method-to-distinguish-direct-dependencies-in-networks}
Recognizing direct relationships between variables connected
in a network is a pervasive problem in biological, social and
information sciences as correlation-based networks contain
numerous indirect relationships. Here we present a general
method for inferring direct effects from an observed
correlation matrix containing both direct and indirect
effects. We formulate the problem as the inverse of network
convolution, and introduce an algorithm that removes the
combined effect of all indirect paths of arbitrary length in
a closed-form solution by exploiting eigen-decomposition
and infinite-series sums. We demonstrate the effectiveness
of our approach in several network applications: distinguishing
direct targets in gene expression regulatory networks; recognizing
directly interacting amino-acid residues for protein structure
prediction from sequence alignments; and distinguishing
strong collaborations in co-authorship social networks using
connectivity information alone. In addition to its theoretical
impact as a foundational graph theoretic tool, our results suggest
network deconvolution is widely applicable for computing direct
dependencies in network science across diverse disciplines.


\cite{Nonparametric-copula-entropy-and-network-deconvolution-method-for-causal-discovery-in-complex-manufacturing-systems}
To clarify the causality among process parameters is a core issue of data-driven production performance analysis and product
quality optimization. The difculty lies in accurately measuring and distinguishing direct and indirect associations of complex
manufacturing systems. In this work, the nonparametric-copula-entropy and network deconvolution method is proposed for
causal discovery in complex manufacturing systems. Firstly, based on copula theory and kernel density estimation method,
the nonparametric-copula-entropy is introduced to improve the accuracy of association measurement between parameters,
and its superiority is verifed by comparing with the results of diferent association measurement methods. Then, the global
association matrix is constructed by the nonparametric-copula-entropy, and network deconvolution method is employed
to extract the direct information from the global association matrix. The proposed method is tested by using an open gene
expression dataset. Finally, as an experimental application, the causal analysis for a diesel engine production line is carried
out by the proposed method. The results show that the proposed method can reveal causal relationship between process
parameters and quality parameters in the diesel engine production line well, which provide theoretical guidance and implementation approach for the optimal control of complex manufacturing system.